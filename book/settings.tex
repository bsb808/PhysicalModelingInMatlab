\usepackage[T1]{fontenc}
\usepackage[utf8]{inputenc}
\usepackage{textcomp} % provide euro and other symbols
\usepackage[table]{xcolor}

\usepackage{roboto}

\usepackage{setspace}
\usepackage{graphicx}
\usepackage{makeidx}
\usepackage{upquote}

\usepackage{enumitem}
\usepackage{textcomp}

% format the table of contents
\usepackage{tocloft}

\usepackage{epigraph} 

% exercise environment
%\usepackage{exercise}
%\renewcommand{\ExerciseHeader}{\textbf{
%\ExerciseName~\ExerciseHeaderNB\ExerciseHeaderTitle
%\ExerciseHeaderOrigin\newline}}
%\newenvironment{ex}{\begin{Exercise}}{\end{Exercise}}

% use amsthm to typeset exercises
\usepackage{amsmath,amsthm,amssymb}
\theoremstyle{definition}
\newtheorem{ex}{Exercise}[chapter]

% set the page size
\usepackage[twoside,centering]{geometry}

% crown quarto
\geometry{
    %showframe=true,
    %showcrop=true,
    % total document size
    papersize={195mm,252mm},
    % book trim size
    layoutsize={189mm,246mm},
    layoutoffset={3mm,3mm},
    % live area = 164mm x 220mm
    % subtract from width for the binding offset
    % subtract from both to improve the look
    total={144mm,200mm},
    bindingoffset=10mm,
    includeheadfoot=true,
    centering=true,
    nofoot=true,
}

% paragraph spacing
\setlength{\parindent}{0pt}
\setlength{\parskip}{12pt plus 4pt minus 4pt}
\linespread{1.05}
\def\arraystretch{1.5}

% list spacing
%\setlength{\topsep}{5pt plus 2pt minus 3pt}      % 10.0pt plus 4.0pt minus 6.0pt
%\setlength{\partopsep}{-6pt plus 2pt minus 2pt}  %  3.0pt plus 2.0pt minus 2.0pt
%\setlength{\itemsep}{0pt}                        %  5.0pt plus 2.5pt minus 1.0pt

% define colors
\usepackage{xcolor}
\definecolor{bgcolor}{HTML}{F0F0F0}
\definecolor{comment}{HTML}{007C00}
\definecolor{keyword}{HTML}{0000FF}
\definecolor{string}{HTML}{B20000}
\definecolor{urlcolor}{HTML}{0000FF}

\definecolor{codeblue}{RGB}{31, 119, 180}
\definecolor{commentblue}{RGB}{31, 119, 180}
\definecolor{codegreen}{RGB}{78,121,53}
\definecolor{codered}{RGB}{214, 39, 40}
\definecolor{codepurple}{RGB}{148, 103, 189}
\definecolor{codegray}{rgb}{0.5,0.5,0.5}
\definecolor{rulecolor}{gray}{0.7}
\definecolor{lightgray}{gray}{0.9}
\definecolor{lightergray}{gray}{0.95}
\usepackage{listings}

\lstset{
    language=matlab,
    basicstyle=\ttfamily,
    backgroundcolor=\color{bgcolor},
    commentstyle=\color{codegreen},
    keywordstyle=\color{codeblue},
    stringstyle=\color{codepurple},
    columns=fullflexible,
    emph={label},  % keyword?
    keepspaces=true,
    showstringspaces=false,
    upquote=true,
    xleftmargin=0pt,  % \parindent
    framexleftmargin=3pt,
    aboveskip=\parskip,
    belowskip=\parskip
}

\lstnewenvironment{code}{}{}
\lstnewenvironment{stdout}{\lstset{commentstyle=,keywordstyle=,stringstyle=}}{}

% inline syntax formatting
\newcommand{\mcode}[1]{\lstinline{#1}}%{

% customize page headers
\usepackage{fancyhdr}
\pagestyle{fancyplain}
\renewcommand{\chaptermark}[1]{\markboth{#1}{}}
\renewcommand{\sectionmark}[1]{\markright{\thesection\ #1}{}}
\lhead[\fancyplain{}{\bfseries\thepage}]%
      {\fancyplain{}{\bfseries\rightmark}}
\rhead[\fancyplain{}{\bfseries\leftmark}]%
      {\fancyplain{}{\bfseries\thepage}}
\cfoot{}
\cfoot{}
%\renewcommand{\headrulewidth}{0pt}
%\rfoot{\textcolor{gray}{\tiny Draft \today}}

% pdf hyperlinks, table of contents, and document properties
\usepackage{hyperref}
\hypersetup{%
  pdftitle={\thetitle},
  pdfauthor={\theauthors},
  pdfsubject={Version \theversion},
  pdfkeywords={},
  bookmarksopen=false,
  colorlinks=true,
  citecolor=black,
  filecolor=black,
  linkcolor=black,
  urlcolor=codepurple
}

% to get siunitx
% sudo apt-get install texlive-science
\usepackage{siunitx}
\sisetup{per-mode=symbol}

\newcommand{\myreg}{\textsuperscript{{\tiny \textregistered}}}

\newcommand{\console}[1]{>{}> \textbf{#1}}
\newcommand{\keycap}{\textsf}

% typesetting vectors
% from https://tex.stackexchange.com/questions/188775/how-to-type-a-particular-kind-of-unit-vector
\usepackage{bm}
\renewcommand{\vec}[1]{\bm{\mathbf{#1}}}
\newcommand{\uveci}{{\bm{\hat{\textnormal{\bfseries\i}}}}}
\newcommand{\uvecj}{{\bm{\hat{\textnormal{\bfseries\j}}}}}
\DeclareRobustCommand{\uvec}[1]{{%
  \ifcsname uvec#1\endcsname
     \csname uvec#1\endcsname
   \else
    \bm{\hat{\mathbf{#1}}}%
   \fi
}}
