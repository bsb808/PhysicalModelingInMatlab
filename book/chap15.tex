\chapter{Under the Hood}
\label{how}


In this chapter we ``open the hood,'' looking more closely at how some of the tools we have used---\lstinline{ode45}, \lstinline{fzero}, and \lstinline{fminsearch}---work.



\section{How fzero Works}
\label{howfzero}

According to the MATLAB documentation, \lstinline{fzero} uses ``a combination of bisection, secant, and inverse quadratic interpolation methods.''  (See \url{https://greenteapress.com/matlab/fzero})

\index{fzero@\lstinline{fzero}}
\index{bisection}
\index{secant method}
\index{inverse quadratic interpolation}
\index{root}

To understand what that means, suppose we're trying to find a root of a function of one variable, $f(x)$, and assume we have evaluated the function at two places, $x_1$ and $x_2$, and found that the results have opposite signs.  Specifically, assume $f(x_1) > 0$ and $f(x_2) < 0$, as shown in Figure~\ref{fig:secant}.

\begin{figure}[h]
\centerline{\includegraphics[scale=0.8]{images/figure15_03_new.eps}}
\caption{Initial state of a root-finding search}
\label{fig:secant}
\end{figure}

As long as $f$ is continuous, there must be at least one root in this interval.
In this case we would say that $x_1$ and $x_2$ \emph{bracket} a zero.

\index{bracket}

If this were all you knew about $f$, where would you go looking for
a root?  If you said ``halfway between $x_1$ and $x_2$,''
congratulations!  You just invented a numerical method called
\emph{bisection}!

If you said, ``I would connect the dots with a straight line
and compute the zero of the line,''
congratulations!  You just invented the \emph{secant method}!

And if you said, ``I would evaluate $f$ at a third point, find the
parabola that passes through all three points, and compute the zeros
of the parabola,'' congratulations, you just invented
\emph{inverse quadratic interpolation}!

That's most of how \lstinline{fzero} works.  The details of how these methods are combined are interesting, but beyond the scope of this book.  You can read more at \url{https://greenteapress.com/matlab/brent}.


\section{How fminsearch Works}
\label{howfminsearch}

According to the MATLAB documentation, \lstinline{fminsearch} uses the Nelder-Mead simplex algorithm.  You can read about it at \url{https://greenteapress.com/matlab/nelder}, but you might find it overwhelming.

\index{fminsearch@\lstinline{fminsearch}}
\index{Nelder-Mead}
\index{golden-section search}

To give you a sense of how it works, I will present a simpler algorithm, the \emph{golden-section search}.  Suppose we're trying to find the minimum of a function of a single variable, $f(x)$.

As a starting place, assume that we have evaluated the function at three places,
$x_1$, $x_2$, and $x_3$, and found that $x_2$ yields the lowest
value. Figure~\ref{fig:golden1} shows this initial state.

\begin{figure}[h]
\centerline{\includegraphics[scale=0.8]{images/figure15_04_new.eps}}
\caption{Initial state of a golden-section search}
\label{fig:golden1}
\end{figure}

We will assume that $f(x)$ is continuous and \emph{unimodal} in this range, which means that there is exactly one minimum between $x_1$ and $x_3$.

\index{minimum}
\index{unimodal}

The next step is to choose a fourth point, $x_4$, and evaluate
$f(x_4)$.  There are two possible outcomes, depending on whether
$f(x_4)$ is greater than $f(x_2)$ or not.
Figure~\ref{fig:golden2} shows the two possible states.

\begin{figure}[h]
\centerline{\includegraphics[scale=0.8]{images/figure15_05_new.eps}}
\caption{Possible states of a golden-section search after evaluating $f(x_4)$}
\label{fig:golden2}
\end{figure}

If $f(x_4)$ is less than $f(x_2)$ (shown on the left), the
minimum must be between $x_2$ and $x_3$, so we would discard $x_1$ and proceed with the new triple $(x_2, x_4, x_3)$.

If $f(x_4)$ is greater than $f(x_2)$ (shown on the right), the
local minimum must be between $x_1$ and $x_4$, so we would discard $x_3$ and proceed with the new triple $(x_1, x_2, x_4)$.

Either way, the range gets smaller and our estimate of the optimal value of $x$ gets better.

This method works for almost any value of $x_4$, but some choices
are better than others.  You might be tempted to bisect the interval between $x_2$ and $x_3$, but that turns out not to be the best choice.  You can read about a better option at \url{https://greenteapress.com/matlab/golden}.

\section{Chapter Review}

The information in this chapter is not strictly necessary; you can use these methods without knowing much about how they work.  But there are two reasons you might want to know.

One reason is pure curiosity.  If you use these methods, and especially if you come to rely on them, you might find it unsatisfying to treat them as ``black boxes.''  At the risk of mixing metaphors, I hope you enjoyed opening the hood.

The other reason is that these methods are not infallible; sometimes things go wrong.  If you know how they work, at least in a general sense, you might find it easier to debug them.

With that, you have reached the end of the book, so congratulations!  I hope you enjoyed it and learned a lot.  I think the tools in this book are useful, and the ways of thinking are important, not just in engineering and science, but in practically every field of inquiry.

Models are the tools we use to understand the world: if you build good models, you are more likely to get things right.  Good luck!

