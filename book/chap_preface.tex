
\chapter*{Preface}

Modeling and simulation are powerful tools for explaining the world, making predictions, designing things that work, and making them work better.  Learning to use these tools can be difficult; this book is my attempt to make the experience as enjoyable and productive as possible.

By reading this book---and working on the exercises---you will learn some programming, some modeling, and some simulation.
With basic programming skills, you can create models for a wide range of physical systems.
My goal is to help you develop these skills in a way you can apply immediately to real-world problems.

This book presents the entire modeling process, including model selection, analysis, simulation, and validation.  I explain this process in Chapter~\ref{modeling}, and there are examples throughout the book.

\section*{Who This Book Is For}

To make this book accessible to the widest possible audience, I've tried to minimize the prerequisites.

This book is intended for people who have never programmed before.  I start from the beginning, define new terms when they are introduced, and present only the features you need, when you need them.

I assume that you know trigonometry and some calculus, but not very much.  If you understand that a derivative represents a rate of change, that's enough.  You will learn about differential equations and some linear algebra, but I will explain what you need to know as we go along.

I will assume you know basic physics, in particular the concepts of force, acceleration, velocity, and position.  If you know Newton's second law of motion in the form $F = m a$, that's enough.


\section*{MATLAB}

This book is based on MATLAB, a programming language originally developed at the University of New Mexico and now produced by MathWorks, Inc.

MATLAB is a high-level language with features that make it well-suited for modeling and simulation, and it comes with a program development environment that makes it well-suited for beginners.

However, one challenge for beginners is that MATLAB uses vectors and matrices for almost everything, which can make it hard to get started.  The organization of this book is meant to help: we start with simple numerical computations, adding vectors in Chapter~\ref{vectors} and matrices in Chapter~\ref{systems}.

Another drawback of MATLAB is that it is ``proprietary''; that is, it belongs to MathWorks, and you can only use it with a license, which can be expensive.

\section*{Working with the Code}

The best way to read this book is with MATLAB open so that you can work the examples as you read.  

The code for each chapter in this book is available on GitHub at \url{https://github.com/bsb808/PhysicalModelingInMatlab/tree/ae2440-dev/code}.  You can download all the code as a ZIP file or you can use the "Download raw file" from a browser to download each file,

in a ZIP file you can download from \url{https://greenteapress.com/matlab/zip}.
Once you have the ZIP file, you can unzip it on the command line by running

I'll provide more information about working with these files when we get to them, but that should be enough to get you started.

If you open any of these files in MATLAB, you should be able to read the code.  To run it press the green \textbf{Run} button.

You might get a message like ``File not found in the current folder.''
MATLAB will give you the option to Change Folder or Add to Path.  If you change folders, you will be able to run this file until you change folders again.  If you add to the path, you will always be able to run this file.

However, as you add more folders to the path, you are more likely to run into problems with name collisions.
I recommend you change folders when necessary and avoid adding folders to the path.

\section*{AE2440 Version}

This `AE2440' version of the book is a customization and extension of Allen Downey's fourth edition of \href{https://greenteapress.com/wp/physical-modeling-in-matlab/}{Physical Modeling in MATLAB}.  I have also used some parts of Professor Downey's 2e edition of \href{https://greenteapress.com/wp/think-python-2e/}{Think Python}.   His free textbooks are very popular and distributed under permissive, open-source licenses.  You can read about why in his \href{https://greenteapress.com/free_books.html}{free textbook manifesto}.

\section*{Contributors}

If you discover an error the AE2440 version this book or the supporting code, or you have suggestions for improving them, please send them to \emph{bbingham@nps.edu}. Or, if you are a GitHub user, you can open an issue or a pull request at \url{https://github.com/bsb808/PhysicalModelingInMatlab}.
    
Special thanks to my collaborators at No Starch Press for their work on this book: Alex Freed, Katrina Taylor, Kelly Kearney, Barbara Yien, Bill Pollock, Richard Hutchinson, and Lisa Devoto Farrell.

Other people who have found errors and helped improve this book include
Michael Lintz,
Kaelyn Stadtmueller,
Roydan Ongie,
Keerthik Omanakuttan,
Pietro Peterlongo,
Li Tao,
Steven Zhang,
Elena Oleynikova,
Kelsey Breseman,
Philip Loh,
Harold Jaffe,
Vidie Pong,
Nik Martelaro,
Arjun Plakkat,
Zhen Gang Xiao,
Zavier Patrick Aguila,
Michael Cline,
Denny Chen,
Matt Wiens,
and Craig Scratchley.

