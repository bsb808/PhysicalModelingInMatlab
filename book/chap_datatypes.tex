\chapter{Data Types}
\label{c:datatypes}


In programming, we work with different kinds of information, or data. A \emph{data type} is a way of classifying this information so that the computer knows how to handle it. Just as different objects in everyday life serve different purposes, various kinds of data are handled differently in a program. For instance, numbers are often used in calculations, while text is treated as characters or words.

Think of data types like containers in your kitchen. If you want to store soup, you’d use a bowl, while juice would go in a glass. Similarly, in programming, you need to choose the right data type to store and process the information correctly. Using the right "container" helps ensure that the computer knows what to do with your data.

In MATLAB, understanding data types is essential not just for organizing your data, but also for interacting with the built-in tools and functions MATLAB offers. These tools use a variety of data types as both input and output. By knowing how to work with different data types, you can ensure that your programs communicate effectively with MATLAB’s powerful built-in functionality.

In this chapter, we'll explore the different data types in MATLAB, when to use them, and how this knowledge will help you interface better with MATLAB's capabilities.