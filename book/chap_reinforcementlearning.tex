\chapter{Reinforcement Learning}

In this chapter, we'll introduce some of the basic concepts of \emph{reinforcement learning} as a primer for implementing a computational example.  Overall, we want to implement a learn-by-doing approach to be able to discover concepts and skills involved in applying machine learning through exploring their implementation; this brief overview provides the basics to start that exploration.

\begin{quote}
Reinforcement learning is learning what to do---how to map situations to actions---so
as to maximize a numerical reward signal. The learner is not told which actions to
take, but instead must discover which actions yield the most reward by trying them. In
the most interesting and challenging cases, actions may affect not only the immediate
reward but also the next situation and, through that, all subsequent rewards. These two
characteristics—trial-and-error search and delayed reward—are the two most important
distinguishing features of reinforcement learning.
\end{quote}
\href{http://www.incompleteideas.net/book/the-book.html}{R. Sutton and A. Barto, “Reinforcement Learning: An Introduction” 2020.}
