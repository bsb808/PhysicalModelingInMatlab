\documentclass[11pt]{article}
%\usepackage[firstpage]{draftwatermark}
\usepackage{times}
\usepackage{pdfpages}
\usepackage{fullpage}
\usepackage{url}
\usepackage{hyperref}
\usepackage{fancyhdr}
\usepackage{graphicx}
\usepackage{tabularx}
\usepackage{enumitem}
\usepackage{indentfirst}
\usepackage{subcaption}
\usepackage{amsmath, amsfonts, amsthm, fouriernc}
\usepackage{units}
\usepackage{IEEEtrantools}

\usepackage{color,soul}
\DeclareRobustCommand{\hlr}[1]{{\sethlcolor{red}\hl{#1}}}
\DeclareRobustCommand{\hlg}[1]{{\sethlcolor{green}\hl{#1}}}
\DeclareRobustCommand{\hlb}[1]{{\sethlcolor{blue}\hl{#1}}}
\DeclareRobustCommand{\hly}[1]{{\sethlcolor{yellow}\hl{#1}}}


\setcounter{secnumdepth}{4}
\graphicspath{{images/}}
\pagestyle{fancy}

\newenvironment{xitemize}{\begin{itemize}\addtolength{\itemsep}{-0.75em}}{\end{itemize}}
\newenvironment{tasklist}{\begin{enumerate}[label=\textbf{\thesubsubsection-\arabic*},ref=\thesubsubsection-\arabic*,leftmargin=*]}{\end{enumerate}}
\newcommand\todo[1]{{\bf TODO: #1}}
\setcounter{tocdepth}{2}
\setcounter{secnumdepth}{4}

\addtolength{\headheight}{2em}
\addtolength{\headsep}{1.5em}
\rhead{ME2801}

\setlength\parindent{0pt}

\begin{document}

\begin{center}
  \Large{\bf Second-Order Model Transfer Function and Impulse Response}
\end{center}

We consider one of the canonical forms of an underdamped second-order system transfer function
\begin{equation}
    G(s) = K_{dc} \frac{\omega_n^2}{(s+\zeta \omega_n)^2 + \omega_d^2}
    \label{e:2nd}
\end{equation}
were $\zeta \omega_n$ is the real part of the complex poles and $\omega_d$ is the imaginary part of the complex poles.
We write it in this form so that the DC-gain of the system is $K_{dc}$.  To illustrate this, recall that $\omega_d = \omega_n \sqrt{1-\zeta^2}$ so that
\begin{equation}
    \lim_{s \rightarrow 0}  [ G(s) ] = K_{dc} \frac{\omega_n^2}{(s+\zeta \omega_n)^2 + \omega_d^2} =K_{dc} \frac{(\zeta\omega_n)^2+\omega_d^2}{(s+\zeta \omega_n)^2 + \omega_d^2} K_{dc}.
\end{equation}
The impulse response of this model is 
\begin{equation}
    c(t) = \mathcal{L}^{-1} [ G(s)],
\end{equation}
We use the Laplace transform pair
\begin{equation}
    \mathcal{L} \left[ e^{-at} \sin{(bt)}\right] = \frac{b}{(s+a)^2 + b^2}
\end{equation}
applied to (\ref{e:2nd}) to solve for
\begin{IEEEeqnarray}{rCl}
    c(t) & = &  \mathcal{L}^{-1} \left[ K_{dc} \left( \frac{\omega_n^2}{\omega_d} \right) 
    \frac{\omega_d}{(s+\zeta \omega_n)^2 + \omega_d^2} \right] \\
     && K_{dc} \left( \frac{\omega_n^2}{\omega_d} \right) e^{-\zeta \omega_n t}
     \sin{(\omega_d t)} \\
     && K_{dc} \left( \frac{(\zeta \omega_n)^2 + \omega_d ^2}{\omega_d} \right) e^{-\zeta \omega_n t}
     \sin{(\omega_d t)}.
\end{IEEEeqnarray}
For example, consider the system modeled by the transfer function
\begin{equation}
    G(s) = K_{dc} \frac{ 0.5^2 + (2 \pi)^2}{(s+0.5)^2 + (2 \pi)^2}
\end{equation}
which has an impulse response 
\begin{equation}
    c(t) = K_{dc} \left(\frac{0.5^2 + (2 \pi)^2}{2 \pi}\right) \, \,e^{-0.5 \, t} \sin((2 \pi) t) 
\end{equation}
\end{document}
