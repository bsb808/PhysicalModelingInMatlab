\documentclass[11pt]{article}
%\usepackage[firstpage]{draftwatermark}
\usepackage{times}
\usepackage{pdfpages}
\usepackage{fullpage}
\usepackage{url}
\usepackage{hyperref}
\usepackage{fancyhdr}
\usepackage{graphicx}
\usepackage{tabularx}
\usepackage{enumitem}
\usepackage{indentfirst}
\usepackage{subcaption}
\usepackage{amsmath, amsfonts, amsthm, fouriernc}
\usepackage{units}
\usepackage{IEEEtrantools}

\usepackage{color,soul}
\DeclareRobustCommand{\hlr}[1]{{\sethlcolor{red}\hl{#1}}}
\DeclareRobustCommand{\hlg}[1]{{\sethlcolor{green}\hl{#1}}}
\DeclareRobustCommand{\hlb}[1]{{\sethlcolor{blue}\hl{#1}}}
\DeclareRobustCommand{\hly}[1]{{\sethlcolor{yellow}\hl{#1}}}



\addtolength{\headheight}{2em}
\addtolength{\headsep}{1.5em}
\rhead{AE2440}

\setlength\parindent{0pt}

\begin{document}

\begin{center}
  \Large{\bf First-Order Response Functions}
\end{center}

In Exercise 4.7 we used a plot to visualize the \emph{free-response} to a first-order, linear ODE.  In this exercise you will write a function to plot the \emph{forced-response} of the system, when the forcing function, the right-hand side of the equation, is constant.  The ODE is 
\[
\dot{y}(t) + (1/\tau) y(t) = f
\]
where $\tau$ is the time constant of the ODE and the constant, non-zero forcing function.  The solution to this ODE is 
\[
y(t) = f(1-e^{-t/\tau}).
\]

Create a live-script named \textbf{firstorder\_function.mlx}.   Add a local function to the live-script that\dots
\begin{itemize}
\item Has three input arguments: $\tau$, $f$ and a label string
\item Adds a plot to the current axes of the solution, $y(t)$, for the given parameters.  The function should plot the solution at 100 evenly spaced points between $t = 0-4\tau$.
The label string should be assigned to the legend display using the \texttt{DisplayName} property - see \url{https://www.mathworks.com/help/matlab/creating_plots/add-legend-to-graph.html}
\item Has two outputs: a vector of the time values ($t$) and a vector of the solution values ($y(t)$) used to generate the plot
\end{itemize}

At this point it is a good idea to test your local function with some nominal values to make sure it works.  Try it with $\tau=1$, $f=1$ and label = \texttt{"hello"} to verify you get a plot as expected.

\vspace{1em}
Now, in the live-script, call the local function to plot the solution equation for three cases: 

\begin{tabular}{|c|c|}
    \hline
    $\tau$ & $f$ \\
    \hline
    0.2   & 3   \\
    \hline
    2   & 2   \\
    \hline
    4   & 1   \\
    \hline
\end{tabular}

The final graph should have a legend and labels with units on the axes.

\end{document}
